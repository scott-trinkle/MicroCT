\documentclass[aps]{article}
\usepackage[letterpaper, margin=1.0in]{geometry} % can add margin=x
\usepackage{amsmath}
\usepackage{amssymb}
\usepackage{float}

\begin{document}

I'm working to get the quantitative basis images from the l-edge data you acquired last week.\newline

My plan is to do a standard decomposition of the projection data into different materials (each with known u/p at the acquisition energies, leaving a weighted sum of density projections). A raw subtraction image, $\Delta g$, of the flat-field corrected sinograms could be modeled as

\begin{align}
  \Delta g = \Delta(\mu/\rho)_{m}A_{m} + \Delta(\mu/\rho)_{o}A_{o}
\end{align}

\noindent where $\Delta(\mu/\rho)$ is the change in mass attenuation coefficient between the two energies, $A$ is a density projection $\left(\int_L \rho(x,y) dl\right)$, ``m'' stands for metal (U or Os) and ``o'' stands for all other materials.

At energies bracketing the L-edges of these metals, we expect $\Delta(\mu/\rho)_m \gg \Delta(\mu/\rho)_o$, so we can model the metal basis images as:

\begin{align}
  A_m \approx \frac{\Delta g}{\Delta (\mu/\rho)_m}
\end{align}

When I do this, there are some negative values in the image, indicating pixels where the other materials are dominating. My overall goal is to use these quantitative density maps to get a better estimate of the number densities in the phase term of the TIE - so negative values aren't helpful.

A more rigorous way would be to break the ``other'' category into further basis materials. 


%We have the following standard x-ray imaging model for a single pixel at single energy $s \in [H,L]$

% \begin{align}
%   I_{m}^{[s]} &= (I_{fl}^{[s]} - I_{d}^{[s]})\text{exp}\left(-\int_L \mu^{[s]}(x,y) dl\right) + I_{d}^{[s]}\\[6pt]
%   I_{cor}^{[s]} &= -\text{ln}\left(\frac{I_m^{[s]}-I_d^{[s]}}{I_{fl}^{[s]}-I_d^{[s]}}\right) = \int_L \mu^{[s]}(x, y) dl
% \end{align}

% \noindent where

% \begin{table}[H]
%   \centering
%   \begin{tabular}{l l}
%     $I_m$: & Raw sample measurement from CCD\\
%     $I_{fl}$: & Flat field measurement\\
%     $I_d$: & Dark field measurement\\
%     $I_{cor}$ & Corrected sample measurement\\
%   \end{tabular}
% \end{table}

% We decompose the attenuation coefficient $\mu$ into $N$ materials:

% \begin{align}
%   I_{cor}^{[s]} &= \int_L \sum_{i=1}^{N} u_i^{[s]}(x,y)dl\\[6pt]
%                 &= \sum_{i=1}^{N} \left(\mu/\rho\right)_i^{[s]}\int_L \rho_i(x,y)dl\\[6pt]
%                 &= \sum_{i=1}^{N} \left(\mu/\rho\right)_i^{[s]} A_i 
% \end{align}

% We are interested in determining $\{A_i\}$, the projections of the mass densities through the sample. My question is: what is the best way to decompose the sample into materials? My assumption would be something like $i \in [\text{H}_2\text{O, U, Os}]$. For the $L$ 

\end{document}